\documentclass[14pt]{beamer}
\usepackage[T2A]{fontenc}
\usepackage[utf8]{inputenc}
\usepackage[russian]{babel}
\usepackage{graphicx}
\usepackage{listings}
\usepackage{xcolor}
\usepackage{tikz}
\usepackage{ulem}

% Настройка подсветки кода
\lstset{
    language=Python,
    basicstyle=\ttfamily\small,
    keywordstyle=\color{blue},
    commentstyle=\color{green},
    stringstyle=\color{red},
    showstringspaces=false,
    breaklines=true,
    frame=single,
    numbers=left,
    numberstyle=\tiny\color{gray},
    tabsize=4
}

\usetheme{Madrid}
\usecolortheme{default}

\title{Библиотека для управления роботом и система турнира по шахматам 5×5}
\institute{Разработка автоматизированных лабораторных работ с элементами соревнования}
\date{\today}

\begin{document}

\begin{frame}
\titlepage
\end{frame}

\begin{frame}
\frametitle{Содержание}
\tableofcontents
\end{frame}

\section{Общее описание проекта}

\begin{frame}
\frametitle{Общее описание проекта}
\begin{itemize}
\item Проект состоит из двух основных компонентов:
\begin{itemize}
\item Библиотека robotlib для управления роботом
\item Система турнира по шахматам 5x5
\end{itemize}
\item Разработано на Python 3
\item Использует современные практики программирования
\end{itemize}
\end{frame}

\section{Библиотека robotlib}

\begin{frame}
\frametitle{Библиотека robotlib}
\begin{block}{Основные команды}
\begin{description}
\item[\texttt{forward()}] Движение вперед на одну клетку
\item[\texttt{backward()}] Движение назад на одну клетку
\item[\texttt{turn\_left()}] Поворот на 90° влево
\item[\texttt{turn\_right()}] Поворот на 90° вправо
\item[\texttt{is\_wall\_*()}] Проверка наличия стены в указанном направлении
\item[\texttt{fill\_area()}] Заливка текущей клетки
\end{description}
\end{block}
\end{frame}

\begin{frame}
\frametitle{Схема действий робота}
\begin{center}
\begin{tikzpicture}[node distance=1.5cm]
\node[draw, rounded corners] (start) {Начало};
\node[draw, rounded corners, below of=start] (fill) {\texttt{fill\_area()}};
\node[draw, rounded corners, below of=fill] (turn) {\texttt{turn\_right()}};
\node[draw, rounded corners, below of=turn] (forward) {\texttt{forward()}};
\node[draw, rounded corners, below of=forward] (end) {Конец};

\draw[->] (start) -- (fill);
\draw[->] (fill) -- (turn);
\draw[->] (turn) -- (forward);
\draw[->] (forward) -- (end);
\end{tikzpicture}
\end{center}
\end{frame}

\begin{frame}
\frametitle{UML-диаграмма классов робота}
\begin{center}
% Замените robot_class_diagram на реальное имя файла с диаграммой
\includegraphics[width=0.8\textwidth]{robot_class_diagram}
%\fbox{\parbox{0.8\textwidth}{\centering Здесь будет UML-диаграмма классов робота}}
\end{center}
\end{frame}

\section{Система турнира по шахматам 5x5}

\begin{frame}[fragile]
\frametitle{Пример работы алгоритма заливки}
\begin{columns}
\column{0.4\textwidth}
\begin{block}{Начальное поле}
\begin{verbatim}
* - # * * * * * * *
* - # * * * * * * *
* - # * * * * * * *
* - # * * * * * * *
* - # * * * * * * *
* - # * * * * * * *
* - # # # # # * * *
* - - - - - - * * *
* - * * * * * * * *
* - * * * * * * * *
\end{verbatim}
\end{block}

\column{0.6\textwidth}
\begin{lstlisting}[basicstyle=\tiny]
def autonomous_filling(self):
    steps = 0
    max_steps = 1000
    last_position = None
    stuck_counter = 0
    
    while self.has_unfilled_cells():
        current_pos = (self.x, self.y)
        
        if current_pos == last_position:
            stuck_counter += 1
            if stuck_counter > 4:
                self.turn_right()
                stuck_counter = 0
        else:
            stuck_counter = 0
            last_position = current_pos
        
        if not self.is_wall_forward():
            self.forward()
        elif not self.is_wall_left():
            self.turn_left()
            self.forward()
        elif not self.is_wall_right():
            self.turn_right()
            self.forward()
        else:
            self.turn_right()
            self.turn_right()
            self.forward()
\end{lstlisting}
\end{columns}
\end{frame}

\begin{frame}
\frametitle{Система турнира по шахматам 5x5}
\begin{block}{Особенности}
\begin{itemize}
\item Доска размером 5x5
\item Фигуры: пешки, ладьи, слоны, короли
\item Алгоритм альфа-бета отсечения
\item Победа при съедении короля
\end{itemize}
\end{block}

\begin{block}{Турнирная система}
\begin{itemize}
\item Двухкруговой турнир
\item Каждая пара играет 2 матча
\item Автоматическое определение победителя
\end{itemize}
\end{block}
\end{frame}

\begin{frame}
\frametitle{Диаграмма последовательности турнира}
\begin{center}
% Замените chess_sequence на реальное имя файла с диаграммой
\includegraphics[width=0.8\textwidth]{chess_sequence}
%\fbox{\parbox{0.8\textwidth}{\centering Здесь будет диаграмма последовательности турнира}}
\end{center}
\end{frame}

\begin{frame}
\frametitle{Диаграмма взаимодействия матчей}
\begin{center}
% Замените tournament_interaction на реальное имя файла с диаграммой
\includegraphics[width=0.8\textwidth]{tournament_interaction}
%\fbox{\parbox{0.8\textwidth}{\centering Здесь будет диаграмма взаимодействия матчей}}
\end{center}
\end{frame}

\begin{frame}[fragile]
\frametitle{Пример записи партии (PGN)}
\begin{lstlisting}
[Event "Турнир 5x5"]
[Site "Локальный"]
[Date "2024.04.01"]
[Round "1"]
[White "Игрок 1"]
[Black "Игрок 2"]
[Result "1-0"]

1. e2-e3 d2-d3
2. Bb1-c2 Bb4-c3
3. Rc1-c2 Rc4-c3
4. Kb2-c2 Kb3-c3
5. c2-c3 c3-c4
6. Bc2-d3 Bc3-d4
7. Rd3-d4 1-0
\end{lstlisting}
\end{frame}

\section{Технические детали}

\begin{frame}
\frametitle{Технические детали}
\begin{table}
\centering
\begin{tabular}{ll}
\textbf{Технология} & \textbf{Назначение} \\
\hline
Python 3 & Основной язык разработки \\
Альфа-бета отсечение & Алгоритм поиска ходов \\
NumPy & Матричные операции \\
\end{tabular}
\caption{Используемые технологии}
\end{table}

\begin{alertblock}{Требования к системе}
\begin{itemize}
\item Python 3.6+
\item Установленные зависимости (pip install -r requirements.txt)
\end{itemize}
\end{alertblock}
\end{frame}

\begin{frame}
\frametitle{Заключение}
\begin{block}{Результаты}
\begin{itemize}
\item Реализована библиотека управления роботом
\item Создана система турнира по шахматам 5x5
\item Разработан алгоритм альфа-бета отсечения
\item Реализована визуализация состояния
\end{itemize}
\end{block}
\end{frame}

\begin{frame}
\begin{center}
\Huge Спасибо за внимание!
\end{center}
\end{frame}

\end{document}